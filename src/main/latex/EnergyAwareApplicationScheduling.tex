
%%%%%%%%%%%%%%%%%%%%%%% file typeinst.tex %%%%%%%%%%%%%%%%%%%%%%%%%
%
% This is the LaTeX source for the instructions to authors using
% the LaTeX document class 'llncs.cls' for contributions to
% the Lecture Notes in Computer Sciences series.
% http://www.springer.com/lncs       Springer Heidelberg 2006/05/04
%
% It may be used as a template for your own input - copy it
% to a new file with a new name and use it as the basis
% for your article.
%
% NB: the document class 'llncs' has its own and detailed documentation, see
% ftp://ftp.springer.de/data/pubftp/pub/tex/latex/llncs/latex2e/llncsdoc.pdf
%
%%%%%%%%%%%%%%%%%%%%%%%%%%%%%%%%%%%%%%%%%%%%%%%%%%%%%%%%%%%%%%%%%%%


\documentclass[a4paper]{article}
\usepackage[utf8]{inputenc}
\usepackage{amssymb}
\usepackage{amsmath}
\usepackage{float}
\usepackage{color}
\usepackage{enumitem} % better description env 


\newcommand\gilles[1]{\textcolor{red}{#1}}
\def\restriction#1#2{\mathchoice
              {\setbox1\hbox{${\displaystyle #1}_{\scriptstyle #2}$}
              \restrictionaux{#1}{#2}}
              {\setbox1\hbox{${\textstyle #1}_{\scriptstyle #2}$}
              \restrictionaux{#1}{#2}}
              {\setbox1\hbox{${\scriptstyle #1}_{\scriptscriptstyle #2}$}
              \restrictionaux{#1}{#2}}
              {\setbox1\hbox{${\scriptscriptstyle #1}_{\scriptscriptstyle #2}$}
              \restrictionaux{#1}{#2}}}
\def\restrictionaux#1#2{{#1\,\smash{\vrule height .8\ht1 depth .85\dp1}}_{\,#2}}

\begin{document}

% first the title is needed
\title{Energy aware scheduling}

\maketitle

\section{Problem formulation}

We consider a problem of scheduling applications in a center with limited power and resources.

This data center runs two different applications : active applications (also called web applications) and batch applications.
Those aplications must be executed on servers, each server having a limited resources availability.

\subsection{Time intervals}

The scheduling is peformed over 24 1-hour long intervals.

Each time interval has its limited power capacity for the center. For this reason we need to consider, on each interval, the actual energy consumed by the applications executed at this interval.

\subsection{Active applications}

We consider $n$ active applications $\{A_{i\in 1..n}\}$, which run continuously over the time intervals.

Each application $A_i$ has 3 different executions modes. Each application's mode produces a given power consumption and a given profit when applied over an interval, and the application must be assigned an execution mode for each time interval, with
\begin{description}[labelwidth=4em,align=left,labelindent=2em]
 \item[$M_{i,j} \in 1..3$] mode of activity $i$ at time $j$.
 \item[$Ea_{i,j} \in 0..1000$] power consumption of activity $i$ at time $j$.
 \item[$Pa_{i,j} \in 0..1000$] profit for running activity $i$ at time $j$.
\end{description}

Those informations can be stored in tables, eg :

\begin{table}[H]
\begin{center}
  \begin{tabular}{ | l | l | l | l |}
     \hline
      & Mode 1 & Mode 2 & Mode 3 \\ \hline
    $A_1$ & 50 & 60 & 70 \\ \hline
    $A_2$ & 40 & 45 & 55 \\
    \hline
    \end{tabular}
\end{center}
\caption{Energy consumption for two active applications}
\end{table}

\begin{table}[H]
\begin{center}
    \begin{tabular}{ | l | l | l | l |}
    \hline
      & Mode 1 & Mode 2 & Mode 3 \\ \hline
    $A_1$ & 100 & 110 & 120 \\ \hline
    $A_2$ & 90 & 100 & 110 \\
    \hline
    \end{tabular}
\end{center}
\caption{Profit for two active applications}
\end{table}

The power and profit of an active application $A_i$ can be comptued with following formula :
\begin{itemize}
 \item element ($M_{i,j}$, $Row^{Energy}_{i}$, $E_{i,j}$) \% \quad $E_i$=$Row^{Energy}_{i} [M_i]$
 \item element ($M_{i,j}$, $Row^{Profit}_{i}$, $P_{i,j}$) \% \quad $P_i$=$Row^{Profit}_{i} [M_i]$
\end{itemize}


\subsection{Batch applications}

We consider $m$ batch jobs $\{B_{i \in1..m}\}$, each with its own specifications :

\begin{description}[labelwidth=4em,align=left,labelindent=2em]
\item [$Duration_i$] $\le24$, the number of intervals this job must be run before being finished,
\item [$Deadline_i$] $\le24$, the interval at which the job must be finished,
\item [$Pbmax_i$] $\in 0..1000$ is the profit earnt only if the whole job is finished before or at deadline,
\item [$Eb_i$] is the power consumed by the job at each interval it is executed.
\end{description}

\begin{table}[H]
\begin{center}
  \begin{tabular}{ | l | l | l | l | l |}
     \hline
          & Duration & Deadline & Profit & Power \\ \hline
    $B_1$ & 3 & 12 & 100 & 60 \\ \hline
    $B_2$ & 4 & 20 & 150 & 80 \\
    \hline
    \end{tabular}
\end{center}
\caption{examples of Batch applications}
\end{table}

Each batch job $B_i$ can be decomposed in as many subjobs $b'_{i,k}$ as its duration, with
\begin{itemize}
\item $k\in1..Duration_i$, 
\item $Duration_{b'_{i,k}} = 1$,
\item $E_{b'_{i,k}} = Eb_i$
\item $S_{i,k}$ the interval at which $b'_{i,k}$ is executed.
\end{itemize}

Since each subjob must be executed over a different time interval,
\[0 < S_{i,1} < S_{i,2}... < S_{i,Duration_i}\]

We note $Bd_i$ the boolean value indicating if $B_i$ meets its deadline.
\[Bd_i \iff S_{i,Duration_i} \leq Deadline_i\]

The energy consumption of batch job $B_i$ at interval $j$, noted $Eb_{i,j}$, can be deduced from the execution of a subjob :
\[\exists! k, S_{i,k} = j \implies Eb_{i,j} = Eb_i\]
\[\nexists k, S_{i,k} = j \implies Eb_{i,j} = 0\]

Finally, the actual profit of the job $i$, noted $Pb_i$, is deduced from the interval of its last subjob :
\[Pb_i = element (S_{i,Duration_i} , [Q_i,..,Q_i,0,..0])\]
 

\subsection{Application placement and migration}

In the context of a data center containing more than one server, the applications must also be placed at each interval on a server.
An application can be migrated from one server to another. We need to consider the extra energy cost induced by migration in our model.

We introduce the placement variables and a migration cost for both active and batch jobs. This cost is added to an interval's energy use when this interval requires the migration of the job.


\subsubsection{Active job migration}

Active jobs are always running on an host. We name

\begin{description}[labelwidth=4em,align=left,labelindent=2em]
	\item[$HA_{i,j}$] the host of the application $A_i$ during interval $j$,
	\item[$PHA_{i,j}$] the cost of moving the application $A_i$ on the interval $j$.
 		\begin{itemize}
			\item If $HA_{i,j-1}$ is defined, i.e. application $A_i$ was running during interval $j-1$, and $HA_{i,j-1} = HA_{i,j}$ then $P(HA_{i,j}) = 0$.
			\item If $HA_{i,j-1}$ is undefined, then $P(HA_{i,j}) = 0$.
			\item If $HA_{i,j-1}$ is defined and $HA_{i,j-1} \neq HA_{i,j}$ then $P(HA_{i,j})$ is a function of $HA_{i,j-1}$ and $HA_{i,j}$.
		\end{itemize}
\end{description}

\subsubsection{Batch job migration}

Batch jobs can be stopped fo as long as required. However, even when stopped they are still present on the server, as their memory state is saved. We name

\begin{description}[labelwidth=4em,align=left,labelindent=2em]
	\item [$HB_{i,j}$] the host of the application $B_i$ during interval $j$,
	\item [$PHB_{i,j}$] the cost of moving the application $B_i$ on the interval $j$.
		\begin{itemize}
			\item If $HB_{i,j-1}$ is defined i.e. application $B_i$ was not yet completed during interval $j-1$, and $HB_{i,j-1} = HB_{i,j}$ then $P(HB_{i,j}) = 0$.
			\item If $HB_{i,j-1}$ is undefined, then $P(HB_{i,j}) = 0$.
			\item If $HB_{i,j-1}$ is defined and $HB_{i,j-1} \neq HB_{i,j}$ then $P(HB_{i,j})$ is a function of $HB_{i,j-1}$ and $HB_{i,j}$.
		\end{itemize}
\end{description}


\subsection{Energy cap}

We  introduce the maximum available energy in the data center at interval $j$ as $Capacity_j$, where $j\in$ [1,24].

In cumulative constraint modeling, the limit of maximum available energy can not be changed over time. We thus introduce fake jobs in each interval $j$ to match with the $Capacity_j$. The maximum capacity is defined as $max (Capacity_{h})$, ${h}\in$ [1,24]. The total energy cost at interval $j$ is 
\[E_j=\sum_{i\in1..n} Ea_{i,j} + \sum_{i\in1..m} Eb_{i,j}+Mig{\sc cost}_j < Capacity_j\]


\subsection{Memory use}

The data center contains $l$ servers $\{S_k\}, k\in 1..l$. Each server $S_k$ has a memory capacity $M_k$ which limits the number of applications this server can execute.

Executing an application $A_i$ or a job $B_i$ on a server $S_k$ consumes a fixed amount of memory 
on the server over the execution interval. Reciprocally, each application must be run on a server at any time and any job executed during an interval must be run on a server.

We note
\begin{description}
\item[$HA_{i,j}$] the host of the application $A_i$ during interval $j$
\item[$HB_{i,j}$] the host of the job $B_i$ during interval $j$
\item[$Load_{k,j}$] the memory load of the server $k$ during interval time $j$
\item[$mem(C)$] the memory use of the job or application C.
\end{description}

then we know that

\begin{equation}
Load_{k,j} = \sum_{A_i}(mem(A_i) if HA_{i,j}=k) + \sum_{B_i}(mem(B_i) if HB_{i,j=k}
\end{equation}


\section{Solution formulation}

A solution to such a problem consists in :
\begin{itemize}
\item $\forall$ Active application $A_i$ and time interval $j$, the execution mode $M_{i,j}$ at which to run the appliction during the interval
\item $\forall$ Batch job $B_i$, a series time interval $S_{i,1}..S_{i, Duration_i}$ at which to execute the subjobs.
\item $\forall$ time interval $j$, $\forall$ server $S_k $, the set of Applications and jobs hosted on the server during the time interval.
\end{itemize}
with respect to the previoulsy specified conditions.


\section{Objective}
Our objective function is to maximize $P$,\\
$P= \sum_{j=1}^{24} \bigg( \sum_{i=1}^{n} {Pa_{i,j}}\bigg)  + \sum_{i=1}^{m} Pb_i$
\end{document}
